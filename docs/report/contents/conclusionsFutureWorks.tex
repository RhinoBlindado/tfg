\chapter{Conclusiones y Trabajos Futuros}

La estimación de la edad es parte esencial del PB y, además, un problema complejo y de gran importancia para el campo de la AF debido a la subjetividad de los métodos manuales empleados actualmente por los expertos actualmente. Este TFG aborda la clasificación automática de los criterios morfológicos utilizados para la estimación de la edad, centrándose en la sínfisis del pubis con el objetivo de obtener un modelo capaz de automatizar la extracción de dichos criterios, y poder asistir al experto humano en la toma de decisiones.

En primer lugar, se realizó un estudio pormenorizado de la literatura relativa a la estimación de edad de restos óseos, centrándose en la sínfisis del pubis, así como del procesado de modelos 3D por medio de DL. De aquí se pudo observar y concluir que actualmente la AF sigue utilizando mayormente el método de Todd \cite{RefWorks:RefID:19-todd1921age} o variantes, con innovaciones tecnológicas limitadas no ligadas a DL. Por parte del procesado de modelos 3D, se observa que no existe un consenso en la mejor forma de representar los datos y, por ende, existen múltiples maneras de representar la información. Para la problemática actual se seleccionan las mallas poligonales puesto que es un formato ubicuo y práctico que es utilizado ampliamente en la informática gráfica y por los propios antropólogos forenses. Por último, se concluye que no existe hasta el momento otra investigación que haya tomado el enfoque novedoso de clasificar automáticamente las características morfológicas del hueso directamente.

Se realizó un análisis de las propuestas metodológicas que hicieran uso de las mallas poligonales. Se trata de una de las representaciones más novedosas por lo que los estudios existentes son limitados. De los mismos, se selecciona MeshCNN y sus variantes, MedMeshCNN y MeshCNN+, como los \textit{frameworks} a utilizar para la creación de un prototipo que procese las mallas 3D de la sínfisis del pubis. En concreto, después de un análisis que tomase en cuenta la complejidad espacial y temporal, se selecciona MeshCNN como el \textit{framework} a utilizar debido a su buen rendimiento en cuanto al tiempo en consecuencia de la planificación temporal de este trabajo y aprovechando que se tuvo acceso a hardware que permite mitigar su alta complejidad espacial.

Centrándose en la creación de prototipos, se necesitó del desarrollo de diversas herramientas para el preprocesado de los datos, que implicó mayoritariamente la reducción de aristas de las mallas 3D con la menor pérdida de información topológica y el sellado de las mismas, tarea que se logró automatizar en gran medida. Tratándose de un prototipo se decidió centrarse en una característica recomendada por los expertos antropólogos por ser relevante y fácil de comprobar a simple vista: el nódulo óseo. Se generaron dos conjuntos balanceados de datos de 50 y 98 muestras en total. Inicialmente, con las 50 muestras se generaron 5 modelos inspirados en una arquitectura utilizada para clasificación de mallas, de estos 5, se obtiene un modelo que posee un 70\% de \textit{accuracy} con una pérdida $L_{CE}$ de 0.621 que tiende a sobreestimar la ausencia del nódulo óseo. Posteriormente, utilizando las 98 muestras se observa que ahora el modelo sobreentrena, obteniendo un 55\% de \textit{accuracy} y 4.428 de $L_{CE}$. Para mitigar este problema, se proponen 32 modelos con diferentes modificaciones en los hiperparámetros para entrenar con dicho conjunto, lográndose al final, obtener 4 modelos que satisfactoriamente reducen el sobreentrenamiento y obtienen un \textit{accuracy} de 70\%, notándose que ahora, en su mayoría los modelos tienden a sobreestimar la presencia del nódulo óseo, obteniendo un \textit{precision} de 0.64 y un \textit{recall} de 0.90, aunque el mejor modelo obtenido posee un balance que le otorga un F1 de 0.70. La observación de los mapas de activación indican que el modelo no se está centrando del todo en la zona del nódulo óseo, mostrando activaciones por la mayoría del hueso y sobre todo en el costado izquierdo. Se concluye que esto ocurre por una falta de datos o por la existencia de otra característica morfológica asociada a la existencia del nódulo óseo. A pesar de este resultado, puede concluirse que el modelo ha logrado aprender y diferenciar una superficie sumamente compleja, indicando el potencial que tiene este enfoque de cara a resolver el problema.

Por lo tanto, se concluye que se han completado satisfactoriamente los objetivos planteados, obteniéndose un modelo capaz de detectar características morfológicas por medio del uso de mallas poligonales de sínfisis del pubis escaneadas. Todo el código se encuentra disponible en el repositorio de GitHub \url{https://github.com/RhinoBlindado/tfg}, a excepción de los escaneos de la sínfisis ya que son datos confidenciales.

Siendo un proyecto utilizando una técnica novedosa y nunca antes utilizada, existen varias ampliaciones lógicas que se pueden realizar al trabajo realizado. Por un lado, probar el modelo obtenido con las otras características del método de Todd de forma individual así como una aproximación multitarea que busque identificar varias características a la vez.

Por otro lado, se pueden explorar otros métodos que procesen modelos 3D. Recientemente se ha publicado otro \textit{framework} denominado MeshNet++ \cite{singh2021meshnet++}, una ampliación de MeshNet \cite{feng2019meshnet} que obtiene resultados comparables o superiores a MeshCNN y, además, mitiga los diversos problemas y restricciones que posee MeshCNN al procesar las mallas.